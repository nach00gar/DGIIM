\documentclass[a4paper,10pt]{article}
\usepackage[utf8]{inputenc}
\usepackage{amsmath}
\usepackage{vmargin}

\setpapersize{A4}
\setmargins{2.5cm}       % margen izquierdo
{0.8cm}                        % margen superior
{16.5cm}                      % anchura del texto
{23.42cm}                    % altura del texto
{10pt}                           % altura de los encabezados
{1cm}                           % espacio entre el texto y los encabezados
{0pt}                             % altura del pie de página
{2cm}                           % espacio entre el texto y el pie de página


\title{Estudio de la diagonalización de una matriz 4x4}
\author{Ignacio Garach}
\date{8 de marzo de 2019}
\begin{document}

\maketitle



$$
A=\begin{pmatrix}
3&3&0&1\\
-1&-1&0&-1\\
1&2&1&1\\
2&4&0&3\\
\end{pmatrix}
$$

Para obtener los valores propios, calculamos las raíces del polinomio característico.

$$
\begin{vmatrix}
3-\lambda &3&0&1 \\
-1&-1-\lambda &0&-1 \\
1&2&1-\lambda &1 \\
2&4&0&3-\lambda \\
\end{vmatrix}=0
$$

En efecto, desarrollando por la tercera columna se tiene:
$$
(1-\lambda)\begin{vmatrix}
3-\lambda &3&1\\
-1&-1-\lambda &-1\\
2&4&3-\lambda\\
\end{vmatrix}=0
$$

$$
P_A(\lambda)=\lambda^4-6\lambda^3+13\lambda^2-12\lambda+4=0
$$

Factorizamos mediante la regla de Ruffini, sabemos que 1 será raíz.
\\
\begin{center}
\begin{tabular}{|c|c|c|c|c|c|}
\hline 
 & 1 & -6 & 13 & -12 & 4 \\
\hline 
1 &  & 1 & -5 & 8 & -4 \\ 
\hline 
 & 1 & -5 & 8 & -4 & 0 \\ 
\hline 
1 &  & 1 & -4 & 4 &  \\ 
\hline 
 & 1 & -4 & 4 & 0 &  \\ 
\hline 
2 &  & 2 & -4 &  &  \\ 
\hline 
 & 1 & -2 & 0 &  &  \\ 
\hline 
\end{tabular}
$$
\lambda_1=1 \hspace*{1cm} a_{\lambda_1}=2 \hspace*{1cm} \lambda_2=2 \hspace*{1cm} a_{\lambda_2}=2
$$
\\
Al tener 2 valores propios con multiplicidad algebraica 2, aún no podemos afirmar que sea diagonalizable. Debemos comprobar que las multiplicidades geométricas coincidan con las algebraicas, en cuyo caso será diagonalizable.
\\
$$
g_{\lambda_1}=4-rango\begin{pmatrix}
2 & 3 & 0 & 1 \\ 
-1 & -2 & 0 & -1 \\ 
1 & 2 & 0 & 1 \\ 
2 & 4 & 0 & 2
\end{pmatrix}=2=a_{\lambda_1}
$$
$$
g_{\lambda_2}=4-rango\begin{pmatrix}
1 & 3 & 0 & 1 \\ 
-1 & -3 & 0 & -1 \\ 
1 & 2 & -1 & 1 \\ 
2 & 4 & 0 & 1
\end{pmatrix}=1\neq a_{\lambda_2} 
$$
\\

Al no coincidir las multiplicidades del segundo valor propio, falla el criterio general de diagonalización, luego A no es diagonalizable.
\end{center}









\end{document}
